\nonstopmode{}
\documentclass[a4paper]{book}
\usepackage[times,inconsolata,hyper]{Rd}
\usepackage{makeidx}
\usepackage[utf8]{inputenc} % @SET ENCODING@
% \usepackage{graphicx} % @USE GRAPHICX@
\makeindex{}
\begin{document}
\chapter*{}
\begin{center}
{\textbf{\huge Package `catalogoUCsBR'}}
\par\bigskip{\large \today}
\end{center}
\inputencoding{utf8}
\ifthenelse{\boolean{Rd@use@hyper}}{\hypersetup{pdftitle = {catalogoUCsBR: An R package for preparing species listings for the Plant Catalog of Units of Brazilian Conservation.}}}{}
\ifthenelse{\boolean{Rd@use@hyper}}{\hypersetup{pdfauthor = {Pablo Melo; Thuane Bochorny; Rafaela Forzza}}}{}
\begin{description}
\raggedright{}
\item[Type]\AsIs{Package}
\item[Title]\AsIs{An R package for preparing species listings for the Plant Catalog of Units of Brazilian Conservation.}
\item[Version]\AsIs{1.0.0}
\item[Date]\AsIs{2023-06-04}
\item[Maintainer]\AsIs{Pablo Melo }\email{pablopains@yahoo.com.br}\AsIs{}
\item[Description]\AsIs{The catalogo_ucs_BR package is designed to convert species occurrence data from the Jabot Geral, Jabot RB, Reflora, SpeciesLink and GBIF data portals into a more understandable format for use in preparing species listings for the Plant Catalog of Units of Brazilian Conservation.
The package provides tools to verify and standardize scientific names of species, join duplicates and select species records as a voucher to compose lists of plant species in UCs in Brazil, promoting the dissemination of knowledge about biodiversity in protected areas.}
\item[License]\AsIs{GPL (>= 2) | file LICENSE}
\item[Encoding]\AsIs{UTF-8}
\item[LazyData]\AsIs{FALSE}
\item[LazyDataCompression]\AsIs{xz}
\item[Roxygen]\AsIs{list(markdown = TRUE)}
\item[RoxygenNote]\AsIs{7.3.1}
\item[Imports]\AsIs{dplyr,
dplyr,
tidyr,
readr,
stringr,
lubridate,
jsonlite,
sqldf,
rvest,
shiny,
shinydashboard,
rhandsontable,
DT,
rhandsontable,
shinyWidgets,
measurements,
downloader}
\item[Remotes]\AsIs{github::pablopains/catalogoUCsBR}
\item[Depends]\AsIs{R (>= 3.5.0)}
\end{description}
\Rdcontents{\R{} topics documented:}
\inputencoding{utf8}
\HeaderA{app\_prepare}{Prepare Plant Catalog of Units of Brazilian Conservation App}{app.Rul.prepare}
%
\begin{Description}
An R app for preparing species listings for the Plant Catalog of Units of Brazilian Conservation.

Remover registros não informativos, sem coletor, numero de coleta, ano e informações de localidade
\end{Description}
%
\begin{Usage}
\begin{verbatim}
app_prepare()
\end{verbatim}
\end{Usage}
%
\begin{Details}
limpar memória

direcionar memória para processamento temporário em disco

carregar funcões para mensurar tempos de processamento

inicar tempo de processamento

carregar pacotes básicos

cerregar funções desenvolvidas pelo CNCFLora

baixar e tabela FB2020 IPT
Rodar somente em atualizações do IPT, aproximadamente 6 horas de processamento.

carregar tabela FB2020 IPT e funções de acesso
conferência taxonômica
carregar informações da espécie
somente aqui encontramos dados de tipo de vegetação conforme FB2020

carregar funções para acesso APIs FB2020 v1 e v2
conferência taxonômica
carregar informações da espécie

carregar pacotes R e funções

criar pasta para salvar raultados do dataset

ultima versao

taxon

Transformação padrão GBIF de híbrido para wcvp

Not found

finalizar tempo de processamento

pano de fundo

Tela junção de resultados

standardize country names and iso 2 iso3 codes

iso2ToIso3 para utilizar em coordinateCleaner

centroids

conferência textual
\end{Details}
%
\begin{Value}
CSV files
\end{Value}
%
\begin{Section}{0 - Preparar ambiente R}
NA
\end{Section}
%
\begin{Section}{Limpeza}
NA
\end{Section}
%
\begin{Section}{Selecionar UC}
NA
\end{Section}
%
\begin{Section}{Install and load packeges to test}
NA
\end{Section}
%
\begin{Section}{Gerar centroides}
NA
\end{Section}
%
\begin{Section}{Preparaçao}
NA
\end{Section}
%
\begin{Section}{Tela APP--}
NA
\end{Section}
%
\begin{Section}{Server}
NA
\end{Section}
%
\begin{Section}{padronizar nome de paises e codigos iso2 e iso3}
NA
\end{Section}
%
\begin{Section}{coordinateCleaner and bdc}
NA
\end{Section}
%
\begin{Section}{Run the application}
NA
\end{Section}
%
\begin{Author}
Pablo Hendrigo Alves de Melo,
Thuane Bochorny \&
Rafaela Forzza
\end{Author}
%
\begin{SeeAlso}
\code{\LinkA{download.file}{download.file}}, \code{\LinkA{aspell}{aspell}}
\end{SeeAlso}
%
\begin{Examples}
\begin{ExampleCode}

app_prepare()

\end{ExampleCode}
\end{Examples}
\inputencoding{utf8}
\HeaderA{app\_review}{Plant Catalog of Units of Brazilian Conservation App}{app.Rul.review}
%
\begin{Description}
An R app for preparing species listings for the Plant Catalog of Units of Brazilian Conservation.
\end{Description}
%
\begin{Usage}
\begin{verbatim}
app_review()
\end{verbatim}
\end{Usage}
%
\begin{Value}
CSV files
\end{Value}
%
\begin{Author}
Pablo Hendrigo Alves de Melo,
Thuane Bochorny \&
Rafaela Forzza
\end{Author}
%
\begin{SeeAlso}
\code{\LinkA{download.file}{download.file}}, \code{\LinkA{aspell}{aspell}}
\end{SeeAlso}
%
\begin{Examples}
\begin{ExampleCode}

load_app_data_review()

\end{ExampleCode}
\end{Examples}
\printindex{}
\end{document}
